La tabla muestra que en Guatemala, la tasa neta de escolaridad en el nivel primario para 2018 de mujeres en Santa Rosa de 49.3, fue la tasa más baja reportada para ese año para mujeres en todos los departamentos. Esta tasa aumentó a 99.6 en 2019, bajó en 2020 a 97.3, y volvió a disminuir a 95.6 en 2021. 

En 2020 se reportó la tasa neta de escolaridad en el nivel primario más baja entre 2018 y 2021 para ambos sexos en todos los departamentos. Dicha tasa correspondió a mujeres en Izabal con 1.0. Los años 2018 y 2019 reportaron 91.7 y 99.3 respectivamente para las mujeres en Izabal, y en 2021 aumentó a 96.9.  