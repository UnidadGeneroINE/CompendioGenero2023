\begin{tabular}[t]{ccccc}
\toprule
\multicolumn{1}{c}{\textbf{ }} & \multicolumn{2}{c}{\textbf{2018}} & \multicolumn{2}{c}{\textbf{2022}} \\
\cmidrule(l{3pt}r{3pt}){2-3} \cmidrule(l{3pt}r{3pt}){4-5}
\textbf{Pueblos} & \textbf{Mujer} & \textbf{Hombre} & \textbf{Mujer} & \textbf{Hombre}\\
\midrule
\cellcolor[HTML]{B6B3FF}{Maya} & \cellcolor[HTML]{B6B3FF}{10.7} & \cellcolor[HTML]{B6B3FF}{23.5} & \cellcolor[HTML]{B6B3FF}{11.6} & \cellcolor[HTML]{B6B3FF}{22.5}\\
Garífuna & 0.0 & 0.1 & 0.1 & 0.1\\
\cellcolor[HTML]{B6B3FF}{Xinka} & \cellcolor[HTML]{B6B3FF}{0.4} & \cellcolor[HTML]{B6B3FF}{1.4} & \cellcolor[HTML]{B6B3FF}{0.4} & \cellcolor[HTML]{B6B3FF}{0.8}\\
Afrodescendiente\footnote{Los resultados de 2018 y 2022 no son comparables debido a que en 2022 se cambió la desagregación de Pueblos de pertencia para incluir a Afrodescendiente/Creole/Afro mestizo.} & N/A & N/A & 0.7 & 1.6\\
\cellcolor[HTML]{B6B3FF}{Ladino} & \cellcolor[HTML]{B6B3FF}{22.8} & \cellcolor[HTML]{B6B3FF}{41.1} & \cellcolor[HTML]{B6B3FF}{24.8} & \cellcolor[HTML]{B6B3FF}{37.2}\\
Extranjero & 0.0 & 0.0 & 0.1 & 0.2\\
\bottomrule
\end{tabular}
