\begin{center}
{\Bold \LARGE PRESENTACIÓN}\\[2cm]
\end{center}

En el marco del cumplimiento de las funciones el Instituto Nacional de Estadística tiene entre sus principales funciones: Recolectar, elaborar y publicar las estadísticas oficiales…”  

Me complace presentar este compendio estadístico que tiene como objetivo principal visibilizar estadísticas de población con enfoque de género y una caracterización en los ámbitos de la educación, salud, economía y laboral, así como la violencia contra las mujeres y un panorama de la participación sociopolítica de las mujeres en el país. Este trabajo representa un esfuerzo colaborativo entre diversas direcciones del INE comprometidas con el desarrollo sostenible e inclusivo en Guatemala, así como las instituciones que proporcionaron la información para que esto fuera posible. 

La información que se presenta en este trabajo es el resultado de un arduo trabajo de recolección y análisis de datos. Cada uno de los datos estadísticos aquí presentados, es una muestra del complejo entramado social, político y económico que caracteriza a Guatemala. 

En este compendio se hace evidente la necesidad de tomar en cuenta las diferencias entre los distintos grupos que conforman la población guatemalteca, reconociendo la diversidad y las desigualdades que aún existen en nuestro país. Además, se pone de manifiesto el importante papel que juegan los distintos pueblos y así como las distinciones entre hombres y mujeres en los distintos ámbitos del país. 

Se espera que este trabajo sea una herramienta útil para las instituciones, organizaciones y personas usuarias que trabajan en la promoción de la igualdad de género y el reconocimiento de la diversidad en Guatemala. Que sirva como base para el diseño de políticas públicas y estrategias que promuevan un desarrollo inclusivo y sostenible en nuestro país. 

\begin{center}
Brenda Izabel Miranda Consuegra\\
\textbf{Gerente}
\end{center}

