El valor de la Tasa Global de Fecundidad (TGF) se interpreta como el número de hijos e hijas, que en promedio tendría una mujer de una cohorte hipotética de mujeres no expuestas al riesgo de muerte desde el inicio hasta el fin del periodo fértil y que, a partir del momento en que se inicia la reproducción están expuestas a las tasas de fecundidad por edad de la población en estudio. En Guatemala de 2018 a 2021 se observa una baja en la tasa global de fecundidad