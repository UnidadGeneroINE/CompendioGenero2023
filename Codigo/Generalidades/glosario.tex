\begin{description}
	\item[Alfabetismo:] Conocimiento básico de la lectura y la escritura.(RAE)
	\item[Carga global de trabajo:] 
	\item[Comunidad lingüística:] Tipo de relación de dependencia que establece la persona con los medios de producción y el empleador. Estas son: empleado(a) del gobierno, empleado(a) privado, jornalero(a) o peón, empleado(a) doméstico(a), trabajador(a) por cuenta propia; patrón(a), empleador(a) o socio(a), trabajador(a) familiar sin pago, trabajador(a) no familiar sin pago. Instituto Nacional de Estadística -INE-. 2018. Resultados del XII Censo Nacional de Población y VII de Vivienda
	\item[Consejos de Desarrollo:] Es el espacio destinado para la organización y coordinación de la administración pública que esta acargo de la formulación de las políticas de desarrollo urbano y rural, así como la de ordenamiento territorial. Congreso de la República de Guatemala. (1985). Costitución Política de la Republica de Guatemala. Artículo 225.
	\item[Dominio de estudio:] Población o grupo a la cual va dirigida la encuesta. INE. 2022. Manual de procesos de la Dirección de Censos y Encuestas, MP-ST-DCE (VERSIÓN 01). Guatemala.
	\item[Dominio rural nacional:] Compuesta por la muestra de sectores de las áreas rurales del país. INE. 2022. ENEI 2021. Guatemala.
	\item[Dominio resto urbano:] En esta medición Integrado por la muestra de los sectores de las áreas urbanas de todos los departamentos de la república exceptuando el departamento de Guatemala. INE. ENEI 2021
	\item[Dominio urbano metropolitano:] Definido como el dominio que integra la muestra de los sectores de las áreas urbanas del departamento de Guatemala. INE. ENEI 2021. Guatemala.
	\item[Edad simple:] Es el intervalo de tiempo que ha vivido una persona desde su nacimiento y que en la práctica, se expresa en la unidad del tiempo más largo que haya cumplido (años, meses, semanas, días), según los casos.	INE. 2016. Manual de procesos departamento de demografía. Guatemala.
	\item[Esperanza de vida:] Es el tipo de trabajo, profesión u oficio que efectuó la persona ocupada, en el período de referencia. En la ENEI, esta variable, corresponde tanto a la actividad principal como a la actividad secundaria.Asimismo, es importante destacar que se clasifica la Ocupación a nivel de Gran Grupo, de acuerdo a la Clasificación Internacional Uniforme de Ocupaciones (CIUO de la OIT, 2008).
	\item[Estado conyugal:] Define la condición de las personas en relación con los derechos y obligaciones establecidos en la ley respecto al matriomonio y uniones de hecho, incluyen por tanto situaciones de hecho y de derecho. INE. 2013.  Compendio estadístico sobre la situación de niñas adolescentes.
	\item[Fecundidad:] Término con el que se expresa la procreación efectiva de los individuos que componen una población, medida por el número de hijos nacidos vivos. Si bien puede estudiarse la fecundidad de la pareja, del hombre o de la mujer, en la práctica se limita por lo general a la de la mujer en edad de concebir (15 a 49 años). INE. 2016. Manual de procesos departamento de demografía. Guatemala.
	\item[Hogar:] Se considera como hogar a la unidad social conformada por una persona o grupo de personas que residen habitualmente en la misma vivienda particular y que se asocian para compartir sus necesidades de alojamiento, alimentación y otras necesidades básicas para vivir. El hogar es el conjunto de personas que viven bajo el mismo techo y comparten al menos los gastos en alimentación. Una persona sola también puede formar un hogar. INE. 2023. ENEI 2022
	\item[Hogar ampliado:] Es el hogar formado por un hogar nuclear más otros parientes (progenitores, suegros (as), nietos (as), hermanos (as) nueras o yernos, etc.).Instituto Nacional de Estadística. 2018. Resultados del XII Censo Nacional de Población y VII de Vivienda
	\item[Hogar compuesto:] Es el hogar constituido por un hogar nuclear o ampliado más personas sin parentesco con el jefe del hogar (no parientes, pensionistas, entre otras).Instituto Nacional de Estadística. 2018. Resultados del XII Censo Nacional de Población y VII de Vivienda
     	\item[Hogar co-residente o corresidente:] Es el hogar integrado por la persona que ejerce la jefatura de hogar más otras personas sin relación de parentesco con ella. Instituto Nacional de Estadística. 2018. Resultados del XII Censo Nacional de Población y VII de Vivienda
        \item[Hogar nuclear:] Este hogar puede estar compuesto por la jefatura de hogar, su cónyuge e hijos (as); solo la jefatura de hogar o cónyuge con hijos (as); o la jefatura de hogar y conyuge sin hijos (hijas). Instituto Nacional de Estadística. 2018. Resultados del XII Censo Nacional de Población y VII de Vivienda
	\item[Hogar unipersonal:] formado por una sola persona. Instituto Nacional de Estadística. 2018. Resultados del XII Censo Nacional de Población y VII de Vivienda.
	\item[Índice de mortalidad femenina:] 
	\item[Jefatura de hogar:] Es la persona que los demas miembros del hogar reconocen como tal y quién toma las decisiones en este. Puede ser hombre o mujer, o aquella persona que tenga la resposabilidad económica del hogar; también puede ser la persona de mayor edad, siempre y cuando sea residente habitual del hogar. Instituto Nacional de Estadística. 2018. Resultados del XII Censo Nacional de Población y VII de Vivienda
	\item[Padrón electoral:]  es el registro de las personas ciudadanas residentes en cada municipio que se hayan inscrito en el Registro de Ciudadanos, que se se identifica con el código del departamento, del municipio y del núcleo poblacional correspondientes. TSE. Ley electoral y de partidos políticos, Artículo 224. Del padrón electoral. República de Guatemala.
	\item[Planifiación familiar:] 
	\item[Población económicamente activa (PEA):] Todas las personas de 15 años o más, que en la semana de referencia realizaron algún tipo de actividad económica, y las personas que estaban disponibles para trabajar y hacen gestiones para encontrar un trabajo. Se incluyen también las personas que durante la semana de referencia no buscaron trabajo activamente por razones de mercado pero estaban dispuestas a iniciar un trabajo de forma inmediata. ENEI 2022
	\item[Población en edad de trabajar (PET):] Según las normas internacionales, es aquella población que está apta, en cuanto a edad para ejercer funciones productivas. Se le denomina también Población en Edad de Trabajar (PET). Esta se subdivide en Población Económicamente Activa (PEA) y Población No Económicamente activa (PNEA). Para efectos de comparabilidad nacional a edad de la PET se toma a partir de los 10 años o más y para el ámbito internacional a partir de 15 años o más. ENEI 2022
	\item[Población ocupada:] Personas de 15 años o más, que durante la semana de referencia hayan realizado durante una hora o un día, alguna actividad económica, trabajando en el período de referencia por un sueldo o salario en metálico o especie o ausentes temporalmente de su trabajo; sin interrumpir su vínculo laboral con la unidad económica o empresa que lo contrata, es decir con empleo pero sin trabajar". ENEI 2022
 	\item[Pueblo:] se refiere a los pueblos maya, Garífuna, Xinca y Ladino, mas la población Afrodescendiente/Creole/Afromestiza, que están reconocidos como aquellos en los cuales se agrupa la población del país. Cada persona de autoidenficó con algún pueblo de pertenencia o se definió como extranjera. Instituto Nacional de Estadística. 2018. Resultados del XII Censo Nacional de Población y VII de Vivienda.
	\item[Sector económico informal:] Son todos aquellos ocupados en las siguientes categorías:
	\begin{itemize}\itemsep -1pt
		\item	Empleadores, empleados y obreros de empresas de menos de 6 personas.
		\item	Todos los trabajadores por cuenta propia o autónoma, excluyendo profesionales y técnicos. 
		\item	Todos los familiares no remunerados
		\item	Ocupados en servicio doméstico
	\end{itemize}
	\item[Tasa de desempleo abierto activo:] La población desempleada abierta activa como proporción de la Población Económicamente Activa.
	\item[Tasa de desempleo oculto:] La población desempleada oculta como proporción de la Población Económicamente Activa más la Población desocupada Oculta.
	\item[Tasa de ocupación bruta:] La Población Ocupada como proporción de la Población en edad de trabajar.
	\item[Tasa de ocupación específica:] La Población Ocupada como proporción de la Población Económicamente Activa.
	\item[Tasa de participación:] La Población Económicamente Activa (PEA) como proporción de la población en edad de trabajar (PET).
	\item[Tasa global de fecundidad :]Número promedio de hijos que tendría una mujer si todas las mujeres sobrevivieran hasta el final de su período fértil.INE. 2016. Manual de procesos departamento de demografía. Guatemala.
	\item[Trabajador por cuenta propia:] Son las personas que trabajan solas o apoyándose exclusivamente con fuerza de trabajo familiar no remunerada.
	\item[Trabajadores agropecuarios:] Son todos aquellos individuos que trabajan en las actividades agrícolas y pecuarias.
	\item[Trabajadores asalariados:] Son aquellas personas que trabajan para un patrón, empresa o negocio, institución o dependencia, regidos por un contrato escrito o de palabra a cambio de un jornal, sueldo o salario. 
	\item[Trabajadores sin pago:] Son las personas que participan en actividades laborales sin percibir a cambio, remuneración monetaria o en bienes.
	\item[Vivienda particular:] Es un recinto de alojamiento o construcción delimitada o separada por paredes y techo(s) con una entrada o acceso independiente, destinado a alojar uno o más hogares o aquél que no está destinado al alojamiento de personas, pero que al momento de la encuesta se encuentra habitado por hogares y grupos de personas que generalmente preparan sus alimentos, comen, duermen y se protegen del clima.
	
\newpage
Se utilizó la siguiente codificación para incluir en los resultados brevemente las categorías ocupacionales de acuerdo a la Clasificación Industrial Internacional Uniforme de todas las Actividades Económicas (CIIU).
%%%%%%%%%%%%%%%%%%%%%%%%%%%%%%%%%%%%%%%%%%%%%%%%%%%%%%%%%%%%%%%%%%%%%
\begin{table}[ht]
\begin{tabular}{|l|l|}
\hline
\multicolumn{1}{|c|}{\textbf{Categoría en resultados}} & \multicolumn{1}{c|}{\textbf{ Cat. de acuerdo a la Clasificación Industrial Internacional de las Actividades Económicas -CIIU-}}                                        \\ \hline
Agricultura                                            & Agricultura, silvicultura y pesca                                                                \\ \hline
Administración pública                                 & Administración pública y defensa; planes de seguridad social de afiliación obligatoria           \\ \hline
Comercio                                               & Comercio al por mayor y al por menor; reparación de los vehículos de motor y de las motocicletas \\ \hline
Comunicación                                           & Información y comunicación                                                                       \\ \hline
Construcción                                           & Construcción                                                                                     \\ \hline
Financieras y de seguros                               & Actividades financieras y de seguros.                                                            \\ \hline
Industrias manufactureras                              & Industrias manufactureras                                                                        \\ \hline
Inmobiliarias                                          & Actividades inmobiliarias                                                                        \\ \hline
Otros servicios                                        & Otras actividades de servicio                                                                    \\ \hline
Profesionales                                          & Profesionales científicos e intelectuales                                                        \\ \hline
\end{tabular}
\caption{Categorías de acuerdo a la Clasificación Industrial Internacional Uniforme de todas las Actividades Económicas -CIIU-.}
\label{tab:categorias}
\end{table}
%%%%%%%%%%%%%%%%%%%%%%%%%%%%%%%%%%%%%%%%%%%%%%%%%%%%%%%%%%%%%%%%%%%%%
\end{description}