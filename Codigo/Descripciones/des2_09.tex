La tabla muestra el número de casos de mortalidad materna según las 7 causas más comunes del 2018 al 2021. 

Para el 2018 las tres causas más comunes de mortalidad materna fue "complicaciones en el trabajo de parto" con 117 casos, "Hipertensión en el embarazo, el parto y el puerperio" con 72 casos y "Otras afecciones obstétricas" con 38 casos. 

Para el 2019 las tres causas más comunes de mortalidad materna fue "complicaciones en el trabajo de parto" con 103 casos, "Hipertensión en el embarazo, el parto y el puerperio" con 39 casos, y 27 casos en "Otras afecciones obstétricas" y "Atención materna relacionada con el feto"

Para el 2020 las tres causas más comunes de mortalidad materna fue "complicaciones en el trabajo de parto" con 101 casos, "Hipertensión en el embarazo, el parto y el puerperio" con 63 casos y "Otras afecciones obstétricas" con 38 casos. 

Para el 2021 las tres causas más comunes de mortalidad materna fue "Otras afecciones obstétricas" con 105 casos, "Complicaciones en el trabajo de parto" con 83 casos e "Hipertensión en el embarazo, el parto y el puerperio" con 53 casos.