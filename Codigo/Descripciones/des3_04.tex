La tasa neta de escolaridad es la relación porcetual entre eñ alumnado de la edad tradicional de completación del nivel educativo sobre el total de la población de esa edad.

La gráfica muestra que en Guatemala, la tasa neta de escolaridad en el nivel primario de mujeres sobrepasó la de los homres en 2020 y 2021. Entre 2018 y 2022, la tasa aumentó en 2.1 puntos porcentuales para mujeres.