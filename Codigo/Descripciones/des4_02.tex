La tasa de participación representa una fracción a la población económicamente activa con edad para trabajar. Se desagregó por sexo y estado conyugal, definiendo el estado conyugal "con pareja" a personas casadas y unidas y "sin pareja" a personas solteras, divorcidas, viudas y separadas.  

La participación económica de las mujeres "sin pareja" en el dominio Urbano Metropolitano superó en 1.2 puntos porcentuales a la participación de mujeres "con pareja". De igual manera, en el Resto Urbano aquellas mujeres "sin pareja" reportaron 0.1 puntos porcentuales más de participación económica que aquellas "con pareja". En el dominio Rural Nacional la participación de mujeres "con pareja" es igual a la de mujeres "sin pareja". 