\begin{center}
{\Bold \LARGE \color{color3} PRESENTACIÓN}\\[2cm]
\end{center}

El Instituto Nacional de Estadística -INE- presenta los principales resultados de la Encuesta Nacional de Empleo e Ingresos -ENEI 2022-. Esta es la primera encuesta con asistencia de herrmientas digitales para el levantamiento de datos, realizado en campo cara a cara, por medio de la técnica \textit{Computer-Assisted Personal Interviewing} -CAPI- por sus siglas en inglés. La exitosa implementación de esta técnica fortalece la Institución en el Sistema Integrado de Encuestas de Hogar. Esto permitió la generación de datos más precisos, redujo el riesgo de errores, disminuyó el tiempo necesario para analizar la base de datos y facilitó la generación ágil de resultados.

Para la ejecución de está encuesta se contemplaron los Objetivos de Desarrollo Sostenible -ODS- como parámetros en el cálculo de algunos indicadores. 

El proceso de levantamiento de datos se realizó durante el período comprendido del 10 de noviembre al 09 de diciembre. Se conformó por 27 grupos de trabajo,  26 en campo y 1 grupo enfocado en el rescate de viviendas y unidades primarias de muestreo.

Los resultados de la encuesta permitirán un amplio análisis de la situación laboral en Guatemala. A la vez, los datos obtenidos son de aporte para la generación de Políticas Públicas y construcción de proyectos para sectores público y privado, universidades, centros de investigación, organizaciones de la sociedad civil, estudiantes, organismos internacionales y la sociedad en general.\\
\\
\\
\begin{center}
Brenda Izabel Miranda Consuegra\\
\textbf{Gerente}
\end{center}

