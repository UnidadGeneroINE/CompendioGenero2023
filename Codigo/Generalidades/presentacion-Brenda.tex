$\ $\\[3cm]
\begin{center}
{\Bold \LARGE PRESENTACIÓN}\\[2cm]
\end{center}

El Instituto Nacional de Estadística -INE- tiene por objeto formular y realizar la política estadística nacional, a la vez, tiene como función ser el ente rector central de información, recolección, elaboración y publicación de datos estadísticos oficiales.  

En el cumplimiento de las funciones mencionadas, como Gerente del INE me complace presentar el compendio estadístico con enfoque de género 2022 que tiene como objetivo visibilizar estadísticas y una caracterización en los ámbitos de: demografía, educación, salud, economía, empleo, violencias contra las mujeres y participación sociopolítica, entre otros temas. 

El presente documento es el resultado de un esfuerzo colaborativo de recolección y análisis estadístico entre diversas fuentes de información del INE pertenecientes al Sistema Estadístico Nacional -SEN- que proporcionaron información estratégica que confluye en las direcciones y dependencias del INE. A las direcciones y representaciones institucionales agradecemos el compromiso y trabajo realizado. A la vez, se convierte en un aporte al diálogo y análisis sobre situación social, económica y política de las mujeres, como una herramienta útil para las instituciones, organizaciones y personas usuarias que trabajan en la promoción de la igualdad de género para la implementación y promoción de políticas y programas públicos y sociales de desarrollo inclusivo y sostenible en Guatemala.\\[3cm]

\begin{center}
\textbf{Brenda Izabel Miranda Consuegra}\\[0.2cm]
Gerente del Instituto Nacional de Estadística -INE-
\end{center}

