\begin{description}
	\item[Área rural:] Se definen a los lugares poblados que se reconocen oficialmente con la categoría de aldeas, caseríos, parajes, fincas, etc., de cada municipio. Incluye a la población dispersa, según Acuerdo Gubernativo del 7 de abril de 1938.
	\item[Área urbana:] Se consideró como área urbana a las ciudades, villas y pueblos (cabeceras departamentales y municipales), así como a aquellos otros lugares poblados que tienen la categoría de colonia o condominio y los mayores de 2,000 habitantes, siempre que en dichos lugares, el 51 por ciento o más de los hogares disponga de alumbrado con energía eléctrica y de agua por tubería (chorro) dentro de sus locales de habitación (viviendas). Al igual que los censos anteriores, se incluyó como área urbana todo el municipio de Guatemala.
	\item[Categoría ocupacional:] Tipo de relación de dependencia que establece la persona con los medios de producción y el empleador. Estas son: empleado(a) del gobierno, empleado(a) privado, jornalero(a) o peón, empleado(a) doméstico(a), trabajador(a) por cuenta propia; patrón(a), empleador(a) o socio(a), trabajador(a) familiar sin pago, trabajador(a) no familiar sin pago.
	
	\item[Condición de actividad:] Clasificación de la población de 15 años o más, en activa e inactiva, de acuerdo con el desempeño o no de una actividad económica o con la búsqueda de ésta, en la semana de referencia.
	
	\item[Grupo de ocupación:] Es el tipo de trabajo, profesión u oficio que efectuó la persona ocupada, en el periodo de referencia. En la ENEI, esta variable, corresponde tanto a la actividad principal como a la actividad secundaria.Asimismo, es importante destacar que se clasifica la Ocupación a nivel de Gran Grupo, de acuerdo a la Clasificación Internacional Uniforme de Ocupaciones (CIUO de la OIT, 2008).
	\item[Hogar:] Para efectos de la ENEI se considera como hogar a la unidad social conformada por una persona o grupo de personas que residen habitualmente en la misma vivienda particular y que se asocian para compartir sus necesidades de alojamiento, alimentación y otras necesidades básicas para vivir. El hogar es el conjunto de personas que viven bajo el mismo techo y comparten al menos los gastos en alimentación. Una persona sola también puede formar un hogar.
	\item[Ingreso medio laboral:] Es la remuneración que en promedio recibe mensualmente un trabajador asalariado o independiente. Se obtiene dividiendo el total de los ingresos percibidos de un mes entre el número de perceptores de dicho mes.
	\item[Ingresos laborales:] Comprende los todos los ingresos provenientes del empleo asalariado más los ingresos relacionados con el empleo independiente por concepto de beneficio o ganancia en la ocupación principal y secundaria.
	\item[Ingresos salariales:] Se consideran en esta categoría los ingresos percibidos por los ocupados con empleo asalariado en concepto de sueldo, salario, jornal y otras prestaciones tanto en la ocupación principal como en la secundaria.
	\item[Población desempleada:] Personas de 15 años o más, que no estando ocupadas en la semana de referencia, están disponibles y buscaron activamente incorporarse a alguna actividad económica en el lapso del último mes. 
	\item[Población desempleada abierta:] Personas de 15 años o más, que sin estar ocupados en la semana de referencia, buscaron activamente un trabajo y tenían disponibilidad inmediata.
	
	\item[Población desempleada abierta aspirante:] Personas de 15 años o más que buscaron trabajo la semana pasada y que recién buscan incorporarse al mercado de trabajo.
	
	\item[Población desempleada abierta cesante:] Personas de 15 años o más, que buscaron trabajo la semana de referencia y tienen experiencia laboral.
	\item[Población desempleada oculta (PDO):] Personas que no tenían trabajo en la semana de referencia, no buscaban pero estarían dispuestos a trabajar bajo determinadas circunstancias.
	
	\item[Población económicamente activa (PEA):] Todas las personas de 15 años o más, que en la semana de referencia realizaron algún tipo de actividad económica, y las personas que estaban disponibles para trabajar y hacen gestiones para encontrar un trabajo. Se incluyen también las personas que durante la semana de referencia no buscaron trabajo activamente por razones de mercado pero estaban dispuestas a iniciar un trabajo de forma inmediata.
	
	\item[Población en edad de trabajar (PET):] Según las normas internacionales, es aquella población que está apta, en cuanto a edad para ejercer funciones productivas. Se le denomina también Población en Edad de Trabajar (PET). Esta se subdivide en Población Económicamente Activa (PEA) y Población No Económicamente activa (PNEA). Para efectos de comparabilidad nacional a edad de la PET se toma a partir de los 10 años o más y para el ámbito internacional a partir de 15 años o más.
	\item[Población no económicamente activa (PNEA):] Comprende a las personas de 15 años o más, que durante el periodo de referencia no tuvieron ni realizaron una actividad económica ni buscaron hacerlo en el último mes a la semana de levantamiento. Las personas menores de 15 años al no cumplir con la edad especificada para la medición de la fuerza de trabajo se consideran como personas no económicas activas.
	\item[Población inactiva plena:] Personas que no tenían trabajo, no buscaban uno y tampoco estaban dispuestos a trabajar.
	
	\item[Población ocupada:] Personas de 15 años o más, que durante la semana de referencia hayan realizado durante una hora o un día, alguna actividad económica, trabajando en el período de referencia por un sueldo o salario en metálico o especie o ausentes temporalmente de su trabajo; sin interrumpir su vínculo laboral con la unidad económica o empresa que lo contrata, es decir con empleo pero sin trabajar".
	\item[Población ocupada plena:] Conjunto de personas que trabajan las jornadas normales de trabajo.
	
	\item[Población que no está en edad de trabajar:] Todas las personas menores de 15 años.
	
	\item[Población subempleada:] Es aquella población de 15 años o más, cuya ocupación es inadecuada, cuantitativa y cualitativamente, respecto a determinadas normas como nivel de ingreso, aprovechamiento de las calificaciones, productividad de la mano de obra y horas trabajadas.
	\item[Población subempleada invisible por calificaciones:] Conjunto de personas que a pesar de trabajar una jornada normal perciben ingresos anormalmente bajos en relación a sus calificaciones.
	
	\item[Población subempleada invisible por ingreso:] Conjunto de personas ocupadas que, a pesar de trabajar una jornada normal o mayor, perciben un ingreso menor a lo establecido.
	
	\item[Población subempleada visible (PSV):] Conjunto de personas que trabajan involuntariamente menos de la jornada normal (40 hrs/semana en el Sector Público, y 48 hrs/semana en el resto de sectores) y que desearían trabajar más horas.
	
	\item[Rama de actividad económica:] Está referida a la actividad económica que realiza, la finca, el negocio, organismo o empresa en la que trabaja la persona ocupada, en el periodo de referencia. Esta variable, corresponde tanto a la actividad principal como a la actividad secundaria. Se clasifican la rama de actividad económica a nivel de Clase, de acuerdo a la Clasificación Internacional Uniforme de todas las Actividades Económicas (CIIU-Rev.4).
	\item[Residencia habitual:] Se entiende por residencia habitual el hogar en donde la persona encuestada se encuentra establecida. De acuerdo al concepto anterior, se consideran como residentes habituales del hogar, a todas las personas que comen y duermen permanentemente en la vivienda donde habitan. El lugar, donde se encuentra el hogar se considera el asiento principal de su familia, del negocio, del trabajo, estudio y de las actividades sociales y económicas de sus miembros.
	Se consideraran residentes habituales del hogar a las personas que al momento de la encuesta se encuentren viviendo en el hogar y que aunque no tengan tres meses de residir en el mismo, no tienen otra residencia habitual, es decir, consideran al hogar encuestado como su residencia habitual.
	\item[Sector económico informal:] Son todos aquellos ocupados en las siguientes categorías:
	\begin{itemize}\itemsep -1pt
		\item	Empleadores, empleados y obreros de empresas de menos de 6 personas.
		\item	Todos los trabajadores por cuenta propia o autónoma, excluyendo profesionales y técnicos. 
		\item	Todos los familiares no remunerados
		\item	Ocupados en servicio doméstico
	\end{itemize}
	
	\item[Tasa de desempleo abierto activo:] La población desempleada abierta activa como proporción de la Población Económicamente Activa.
	
	\item[Tasa de desempleo oculto:] La población desempleada oculta como proporción de la Población Económicamente Activa más la Población desocupada Oculta.
	
	\item[Tasa de ocupación bruta:] La Población Ocupada como proporción de la Población en edad de trabajar.
	
	\item[Tasa de ocupación específica:] La Población Ocupada como proporción de la Población Económicamente Activa.
	
	\item[Tasa de participación:] La Población Económicamente Activa (PEA) como proporción de la población en edad de trabajar (PET).
	
	\item[Trabajador por cuenta propia:] Son las personas que trabajan solas o apoyándose exclusivamente con fuerza de trabajo familiar no remunerada.
	
	\item[Trabajadores agropecuarios:] Son todos aquellos individuos que trabajan en las actividades agrícolas y pecuarias.
	
	\item[Trabajadores asalariados:] Son aquellas personas que trabajan para un patrón, empresa o negocio, institución o dependencia, regidos por un contrato escrito o de palabra a cambio de un jornal, sueldo o salario. 
	
	\item[Trabajadores sin pago:] Son las personas que participan en actividades laborales sin percibir a cambio, remuneración monetaria o en bienes.
	
	\item[Vivienda particular:] Es un recinto de alojamiento o construcción delimitada o separada por paredes y techo(s) con una entrada o acceso independiente, destinado a alojar uno o más hogares o aquél que no está destinado al alojamiento de personas, pero que al momento de la encuesta se encuentra habitado por hogares y grupos de personas que generalmente preparan sus alimentos, comen, duermen y se protegen del clima.
	
\newpage
Se utilizó la siguiente codificación para incluir en los resultados brevemente las categorías ocupacionales de acuerdo a la Clasificación Industrial Internacional Uniforme de todas las Actividades Económicas (CIIU).
%%%%%%%%%%%%%%%%%%%%%%%%%%%%%%%%%%%%%%%%%%%%%%%%%%%%%%%%%%%%%%%%%%%%%
\begin{table}[ht]
\begin{tabular}{|l|l|}
\hline
\multicolumn{1}{|c|}{\textbf{Categoría en resultados}} & \multicolumn{1}{c|}{\textbf{ Cat. de acuerdo a la Clasificación Industrial Internacional de las Actividades Económicas -CIIU-}}                                        \\ \hline
Agricultura                                            & Agricultura, silvicultura y pesca                                                                \\ \hline
Administración pública                                 & Administración pública y defensa; planes de seguridad social de afiliación obligatoria           \\ \hline
Comercio                                               & Comercio al por mayor y al por menor; reparación de los vehículos de motor y de las motocicletas \\ \hline
Comunicación                                           & Información y comunicación                                                                       \\ \hline
Construcción                                           & Construcción                                                                                     \\ \hline
Financieras y de seguros                               & Actividades financieras y de seguros.                                                            \\ \hline
Industrias manufactureras                              & Industrias manufactureras                                                                        \\ \hline
Inmobiliarias                                          & Actividades inmobiliarias                                                                        \\ \hline
Otros servicios                                        & Otras actividades de servicio                                                                    \\ \hline
Profesionales                                          & Profesionales científicos e intelectuales                                                        \\ \hline
\end{tabular}
\caption{Categorías de acuerdo a la Clasificación Industrial Internacional Uniforme de todas las Actividades Económicas -CIIU-.}
\label{tab:categorias}
\end{table}
%%%%%%%%%%%%%%%%%%%%%%%%%%%%%%%%%%%%%%%%%%%%%%%%%%%%%%%%%%%%%%%%%%%%%
\end{description}